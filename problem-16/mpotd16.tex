\documentclass[12pt]{article}
\usepackage[framemethod=TikZ]{mdframed}
\usepackage{amsthm}
\usepackage{tcolorbox}
\usepackage[margin=0.75in]{geometry}
\usepackage{setspace}
\usepackage{amsmath}
\usepackage{cancel}
\usepackage{multicol}
\def\deg{\ensuremath{^\circ}}
\def\v#1{\ensuremath{\mathrm{#1}}}
\usepackage{multirow}
\usepackage{fancyhdr}
\pagestyle{fancy}
\usepackage{lastpage} 
\usepackage{colortbl}
\usepackage{amssymb}
\usepackage{graphicx}
\setlength{\parskip}{0pt}
\lhead{Jhon Christian N. Rozano}
\chead{}
\rhead{May 31, 2021}

\lfoot{Source: Brilliant}
\cfoot{}
\rfoot{Page \thepage\ of \pageref{LastPage}}

\renewcommand{\headrulewidth}{1pt}

\renewcommand{\footrulewidth}{1pt}

\newcommand*\Eval[3]{\left.#1\right\rvert_{#2}^{#3}}

\definecolor{LightCyan}{rgb}{0.88,1,1}

\renewcommand{\familydefault}{\sfdefault}


\begin{document}
	\setlength{\columnsep}{10pt}
	\renewcommand{\arraystretch}{1.5}
	\singlespacing
	\newtcolorbox{mybox}[1]{title=#1}
	
	\begin{center}
		{\large  \textbf{Physics Problems of the Day}} \\ 
	\end{center}
	\textbf{Problem 1:} On a school trip, Otto tries to take a selfie with the bottom of a deep well, but the phone slips out of his hand and into the well. He drops the phone \( 10\text{ m}  \) above the water's surface, and he listens for the splash from the same height. How many seconds will it take before Otto hears the splash? \\
	\textbf{Note:} Assume no air resistance, \( g = 9.81{\text{ m/s}}^2 \), and the speed of sound equal to \(343\text{ m/s}\).
	
	\begin{mybox}{\textbf{Solution}}
		Let $t$ be the time it takes to travel $10\text{ m}$ below the well. Then applying kinematics equation, we get
		\[ \Delta x = \frac{1}{2}at^2 \Rightarrow 10 = \frac{1}{2}\times 9.81\text{ m/s}^2 \times t^2  \]
		Therefore, \( t \approx 1.43\text{ s} \). After the phone drops into the well, the sound will also travel \(10\text{ m} \), reaching Otto's ears. Since the time is equal to distance divided by speed, we get
		\[ t_1 = \frac{\text{distance}}{\text{speed}} = \frac{10\text{ m}}{343\text{ m/s}} \approx 0.03\text{ s}\]
		Thus, the total time it will take before Otto hears the splash is 
		\[ t_\text{total} = 1.43\text{ s} + 0.03\text{ s} = \boxed{1.46\text{ s}} \]
	\end{mybox}
	\medskip
	\noindent \textbf{Problem 2:} The position of a $3\text{ kg}$ particle moving on the
	\( xy-\text{plane} \) at time $t$  (in seconds) is given by the equation \( \vec{r} = \left( 2t^2 \hat{i} - \left( 4t+ 4t^2 \right) \hat{j} \right) \) m. Considering the origin $O$ as the pivot, what is the magnitude of the torque on the particle at \( t = 6\text{ s} \)?
	\begin{mybox}{\textbf{Solution}}
		The formula for torque is given by $\vec{\tau} = \vec{r} \times \vec{F} $. We need to look for $\vec{F}$ and $\vec{r}$. At \( t = 6\text{ s} \) the particle travels \[ \vec{r} = \left( (2 \times 6^2)\hat{i} - ( 4 \times 6 + 4 \times 6^2) \hat{j}    \right) = 72 \hat{i} - 168 \hat{j}  \]
		
		Then, to compute the force exhibited by the particle, we know that
		\[ \vec{a} = \dfrac{\mathrm{d}\vec{v}}{\mathrm{d}t} =  \dfrac{\mathrm{d^2}\vec{x}}{\mathrm{d}t^2} = \dfrac{\mathrm{d^2}\vec{r}}{\mathrm{d}t^2} \Rightarrow \vec{\tau} = \vec{r} \times \vec{F} = \vec{r} \times m \cdot \dfrac{\mathrm{d^2}\vec{r}}{\mathrm{d}t^2}  \] 
		\[\vec{a} = \dfrac{\mathrm{d^2}\vec{r}}{\mathrm{d}t^2} =  \dfrac{\mathrm{d^2}}{\mathrm{d}t^2} \left( 2t^2 \hat{i} - \left( 4t+ 4t^2 \right) \hat{j} \right)\text{ m} = \left( 4\hat{i} - 8\hat{j} \right)\text{ m/s}^2   \]
		Thus, the force of the particle at \( t = 6\text{ s} \) is 
		\[ \vec{F} = m \cdot \vec{a} = \left( 3\text{ kg} \right) \left( 4\hat{i} - 8\hat{j} \right)\text{ m} = \left( 12\hat{i} - 24\hat{j} \right)\text{ N} \]
		Therefore, the torque \(\tau \) is 
		\[ \vec{\tau} = \vec{r} \times \vec{F} = \left( 72 \hat{i} - 168 \hat{j} \right) \times \left(  12\hat{i} - 24\hat{j}  \right)\text{N}\cdot\text{m} = -1728 \hat{k} + 2016\hat{k} \text{ N}\cdot\text{m} = 288\hat{k} \text{ N}\cdot\text{m} \]
		The magnitude of the torque is \( \boxed{ \vec{\tau} = 288\text{ N} \cdot \text{m}  }\)
		
	\end{mybox}
	
	
\end{document}
