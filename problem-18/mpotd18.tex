\documentclass[12pt]{article}
\usepackage[framemethod=TikZ]{mdframed}
\usepackage{amsthm}
\usepackage{tcolorbox}
\usepackage[margin=0.75in]{geometry}
\usepackage{setspace}
\usepackage{amsmath}
\usepackage{cancel}
\usepackage{multicol}
\def\deg{\ensuremath{^\circ}}
\def\v#1{\ensuremath{\mathrm{#1}}}
\usepackage{multirow}
\usepackage{fancyhdr}
\pagestyle{fancy}
\usepackage{lastpage} 
\usepackage{array, caption, tabularx,  ragged2e}
\usepackage{colortbl}
\usepackage{amssymb}
\usepackage{graphicx}
\usepackage[parfill]{parskip}
\setlength{\parskip}{0pt}
\lhead{Jhon Christian N. Rozano}
\chead{}
\rhead{July 5, 2021}

\lfoot{Source: Brilliant}
\cfoot{}
\rfoot{Page \thepage\ of \pageref{LastPage}}

\renewcommand{\headrulewidth}{1pt}

\renewcommand{\footrulewidth}{1pt}

\newcommand*\Eval[3]{\left.#1\right\rvert_{#2}^{#3}}

\definecolor{LightCyan}{rgb}{0.88,1,1}

\renewcommand{\familydefault}{\sfdefault}
\def\mathLarge#1{\mbox{\LARGE $#1$}}

\begin{document}
	\setlength{\columnsep}{10pt}
	\renewcommand{\arraystretch}{1.5}
	\singlespacing
	\newtcolorbox{mybox}[1]{title=#1}
	\begin{center}
		{\large  \textbf{Math Problems of the Day}} \\ 
	\end{center}

	\medskip
	
	\textbf{Problem 1:} Is the following a \textbf{tautology}, a \textbf{contradiction}, or a \textbf{contingent}?
	\begin{equation*}
	\mathLarge{\neg \left(  A \wedge \left(  \neg B \right) \right) \leftrightarrow \left(A \rightarrow B \right)} 
	\end{equation*}
	
	\begin{mybox}{\textbf{Solution}}
		We can use the truth table to determine if it is tautology, contradiction or contingent. \\
		
		\begin{center}
		\begin{tabular}{|c|c|c|c|c|c|c|c|}
			\hline
			$A$ & $B$ & $\neg B$ & $A \wedge (\neg B)$ & $\neg (A \wedge (\neg B))$  & $A \rightarrow B$ & $\neg \left(  A \wedge \left(  \neg B \right) \right) \leftrightarrow \left(A \rightarrow B \right)$ \\
			\hline
			1 & 1 & 0 & 0 & 1 & 1 & 1 \\
			\hline
			0 & 1 & 0 & 0 & 1 & 1 & 1 \\
			\hline
			1 & 0 & 1 & 1 & 0 & 0 & 1 \\
			\hline
			0 & 0 & 1 & 0 & 1 & 1 & 1 \\
			\hline
		\end{tabular}
		\end{center}
	
		\medskip
		Since all of the values in the last column are true, the statement $ \neg \left(  A \wedge \left(  \neg B \right) \right) \leftrightarrow \left(A \rightarrow B \right) $ is indeed a tautology.
	\end{mybox}

	\medskip
	
	\textbf{Problem 2:} Show that this proposition is a \textbf{tautology}:
	\begin{equation*}
		\mathLarge{((A \wedge B) \rightarrow C ) \leftrightarrow (A \rightarrow (B \rightarrow C))} 
	\end{equation*}
	
		\begin{mybox}{\textbf{Solution}}
		We can use the truth table to show that the statement is tautology. \\
		
		\begin{center}
			\begin{tabularx}{\linewidth}{|c|c|c|c|c|c|c|*{2}{>{\RaggedRight\arraybackslash}X|}|}
				\hline
				$A$ & $B$ & $C$ & $A \wedge B$  & $(A \wedge B) \rightarrow C$  & $B \rightarrow C$  & $A \rightarrow (B \rightarrow C)$ & $((A \wedge B) \rightarrow C ) \leftrightarrow (A \rightarrow (B \rightarrow C))$  \\
				\hline
				1 & 1 & 1 & 1 & 1 & 1 & 1 & 1 \\
				\hline
				1 & 1 & 0 & 1 & 0 & 0 & 0 & 1 \\
				\hline
				0 & 1 & 1 & 0 & 1 & 1 & 1 & 1 \\
				\hline
				1 & 0 & 1 & 0 & 1 & 1 & 1 & 1 \\
				\hline
				0 & 0 & 1 & 0 & 1 & 1 & 1 & 1 \\
				\hline
				0 & 1 & 0 & 0 & 1 & 0 & 1 & 1 \\
				\hline
				1 & 0 & 0 & 0 & 1 & 1 & 1 & 1 \\
				\hline
				0 & 0 & 0 & 0 & 1 & 1 & 1 & 1 \\
				\hline
			\end{tabularx}
		\end{center}
		
		\medskip
		Since all of the values in the last column are true, the statement $ ((A \wedge B) \rightarrow C ) \leftrightarrow (A \rightarrow (B \rightarrow C)) $ is therefore a tautology. \qedsymbol
	\end{mybox}
	
	
\end{document}
