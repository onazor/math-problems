\documentclass[14pt]{extreport}
\usepackage[framemethod=TikZ]{mdframed}
\usepackage{amsthm}
\usepackage{tcolorbox}
\usepackage[margin=0.75in]{geometry}
\usepackage{setspace}
\usepackage{amsmath}
\usepackage{cancel}
\usepackage{multicol}
\def\deg{\ensuremath{^\circ}}
\def\v#1{\ensuremath{\mathrm{#1}}}
\usepackage{multirow}
\usepackage{fancyhdr}
\pagestyle{fancy}
\usepackage{lastpage} 
\usepackage{array, caption, tabularx,  ragged2e}
\usepackage{colortbl}
\usepackage{amssymb}
\usepackage{graphicx}
\usepackage[parfill]{parskip}
\setlength{\parskip}{0pt}
\lhead{Jhon Christian N. Rozano}
\chead{}
\rhead{July 8, 2021}

\lfoot{Source: Brilliant}
\cfoot{}
\rfoot{Page \thepage\ of \pageref{LastPage}}

\renewcommand{\headrulewidth}{1pt}

\renewcommand{\footrulewidth}{1pt}

\newcommand*\Eval[3]{\left.#1\right\rvert_{#2}^{#3}}

\definecolor{LightCyan}{rgb}{0.88,1,1}

\renewcommand{\familydefault}{\sfdefault}
\def\mathLarge#1{\mbox{\LARGE $#1$}}

\begin{document}
	\setlength{\columnsep}{10pt}
	\renewcommand{\arraystretch}{1.5}
	\singlespacing
	\newtcolorbox{mybox}[1]{title=#1}
	\begin{center}
		{\large  \textbf{Math Problem of the Day}} \\ 
	\end{center}
	\medskip
	\noindent
	Let $\max (x,y)$ and $\min (x,y)$ be defined as follows for real numbers $x$ and $y$:
	\[ \max (x,y) = \begin{cases}
					x & (x \geq y) \\
					y & (x < y)  \\
					\end{cases} 
	\]
	\[ \min (x,y) = \begin{cases}
					x & (x < y) \\
					y & (x \geq y)  \\
					\end{cases} 
	\]
	If the following holds for distinct numbers $x$ and $y$, what is $3xy$:
	\[ \max (x,y) = 5x -2y + 79 \hspace{3em} \min (x,y) = 4x + 3y -47 \]
	\begin{mybox}{\textbf{Solution}}
		WLOG, let $x > y$, then the functions will be
		\[ \max (x,y) = x \hspace{3em} \min(x,y) = y \]
		From this, we get the following equations:
		\begin{equation}\label{max}
			\max (x,y) = 5x - 2y + 79 = x \Rightarrow 4x - 2y = -79 \\
		\end{equation}
		\begin{equation}\label{min}
			\min (x,y) = 4x + 3y + -47 = y \Rightarrow 4x + 2y = 47 \\
		\end{equation} 
	Solving for the values of $x$ and $y$ using equations \eqref{max} and \eqref{min}, we get $(x,y) = (-4, \frac{63}{2})$. However, this contradicts the assumption that $x > y$. \\
	\medskip
	
	\noindent
	Let $x < y$, then we get
		\[ \max (x,y) = y \hspace{3em} \min(x,y) = x \]
	Substituting these to the equations from the given, we have
			\begin{equation}\label{max1}
			\max (x,y) = 5x - 2y + 79 = y \Rightarrow 5x - 3y = -79 \\
		\end{equation}
		\begin{equation}\label{min2}
			\min (x,y) = 4x + 3y + -47 = x \Rightarrow 3x + 3y = 47 \\
		\end{equation} 
	Solving for the values of $x$ and $y$ using equations \eqref{max1} and \eqref{min2}, we get $(x,y) = (-4, \frac{59}{3})$. This satisfies $ x = -4 < y = \frac{59}{3}$, thus the value of $3xy$ is equal to $3(-4)\left( \frac{59}{3} \right) = \boxed{\textcolor{red}{-236}} $
	\end{mybox}
\end{document}
