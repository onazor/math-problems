\documentclass[12pt]{article}
\usepackage[framemethod=TikZ]{mdframed}
\usepackage{amsthm}
\usepackage{tcolorbox}
\usepackage[paperheight=13in, paperwidth=8.5in, margin=0.75in]{geometry}
\usepackage{setspace}
\usepackage{amsmath}
\usepackage{cancel}
\usepackage{multicol}
\def\deg{\ensuremath{^\circ}}
\def\v#1{\ensuremath{\mathrm{#1}}}
\usepackage{multirow}
\usepackage{fancyhdr}
\pagestyle{fancy}
\usepackage{lastpage} 
\usepackage{array, caption, tabularx,  ragged2e}
\usepackage{colortbl}
\usepackage{amssymb}
\usepackage{graphicx}
\usepackage[parfill]{parskip}
\setlength{\parskip}{0pt}
\lhead{Jhon Christian N. Rozano}
\chead{}
\rhead{July 19, 2021}

\lfoot{Source: Brilliant}
\cfoot{}
\rfoot{Page \thepage\ of \pageref{LastPage}}

\renewcommand{\headrulewidth}{1pt}

\renewcommand{\footrulewidth}{1pt}

\newcommand*\Eval[3]{\left.#1\right\rvert_{#2}^{#3}}

\definecolor{LightCyan}{rgb}{0.88,1,1}

\renewcommand{\familydefault}{\sfdefault}
\def\mathLarge#1{\mbox{\LARGE $#1$}}

\begin{document}
	\setlength{\columnsep}{10pt}
	\renewcommand{\arraystretch}{1.5}
	\singlespacing
	\newtcolorbox{mybox}[1]{title=#1}
	
	\begin{center}
		{\large  \textbf{Physics Problems of the Day}} \\
	\end{center}

	\medskip
	
	\begin{enumerate}
		\item Samira launches a mini-rocket from the ground. The rocket immediately expends all of its fuel and launches vertically with initial speed $30\text{ m/s}$. When the rocket reaches its highest point, it has lost $40\%$ of the initial energy due to air resistance. \\
		
		Assuming $g = 10\text{ m/s}^2$, how much higher would the rocket have gone if there was no air resistance?
		
		\begin{mybox}{\textbf{Solution}}
			Note that if there is no air resistance, the final kinetic energy of the mini-rocket at its highest point of reach is equal to its initial potential energy. Thus,
			\[ mgh = \frac{1}{2} mv^2 \Rightarrow h = \frac{v^2}{2g} \]
			Plugging in the values above, we get
			\[ h = \frac{(30\text{ m/\cancel{s}})^{\cancel{2}}}{2(10\text{ \cancel{m/s}}^{2})} = 45\text{ m}\]
			However, there is an air resistance present in the system. Since $40\%$ of the energy has lost at its highest point, we can denote that
			\[ mgh_1 = \left( 1-0.4 \right)\frac{1}{2}mv^2 \]
			Plugging in the values, we have
			\[ h_1 = 0.3 \times \frac{(30\text{ m/\cancel{s}})^{\cancel{2}}}{10\text{ \cancel{m/s}}^2} = 27\text{ m} \]
			Therefore, if there is no air resistance present in the system, the mini-rocket would take $45\text{ m}-27\text{ m} = \boxed{18\text{ m}}$ more to travel.
		\end{mybox}
	
		\vspace{0.5cm}
	
		\item Esme and Fitz start driving their cars simultaneously, both with constant acceleration. The ratio of Fitz's acceleration to Esme's is $\frac{2}{3}$, and Esme's trip takes twice as much time as Fitz's. What is the ratio of Fitz's travel distance to Esme's?
			\begin{mybox}{\textbf{Solution}}
				Note that the distance formula is given by
				\[ x = \frac{1}{2}at^2 \]
				where $x$, $a$, and $t$ denote the distance, acceleration, and time. Let $x_F$ and $x_E$ be the distance travelled by Fitz and Esme, respectively; $a_F$ and $a_E$ be the acceleration; $t_F$ and $t_E$ be the time. Thus, we get 
				\begin{align*}
					\frac{x_F}{x_E} = & \frac{0.5 \times a_F \times {t_F}^2 }{0.5 \times a_E \times {t_E}^2} \\
					= & \frac{a_F}{a_E} \times \left( \frac{t_F}{t_E} \right)^2 \\
					\frac{x_F}{x_E} = & \frac{2}{3} \times \left( \frac{1}{2} \right)^2 = \frac{1}{6}
				\end{align*}
				Therefore, the ratio of Fitz's travel distance to Esme's is $\boxed{\frac{1}{6}}$.
			\end{mybox}
	\end{enumerate}
	

\end{document}
