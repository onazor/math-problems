\documentclass[12pt]{article}
\usepackage[letterpaper, margin=1in]{geometry}
\usepackage{setspace}
\usepackage[dvipsnames]{xcolor}
\usepackage{ragged2e}
\usepackage{amsmath}
\usepackage{cancel}
\usepackage{multicol}
\usepackage{multirow}
\usepackage{fancyhdr}
\pagestyle{fancy}
\usepackage{lastpage} 
\usepackage{xcolor,colortbl}
\usepackage{amssymb}

\lhead{March 31, 2021}
\chead{}
\rhead{jnrozano@up.edu.ph}

\lfoot{Jhon Christian N. Rozano}
\cfoot{}
\rfoot{Page \thepage\ of \pageref{LastPage}}

\renewcommand{\headrulewidth}{1pt}

\renewcommand{\footrulewidth}{1pt}

\newcommand*\Eval[3]{\left.#1\right\rvert_{#2}^{#3}}

\definecolor{LightCyan}{rgb}{0.88,1,1}

\begin{document}
	\setlength{\columnsep}{20pt}
	\renewcommand{\arraystretch}{1.5}
	\singlespacing
	
	\noindent
	\begin{center}
	{\large  \textbf{Math Problems of the Day}} \\ 
 	NYCIML Spring 1975
	\end{center}

	\noindent
	1. Find all positive numbers $x$ that satisfy
	\[ {\left( 2 + \log x \right)}^3 + {\left( -1 + \log x \right)}^3 = {\left( 1 + \log x^2 \right)}^3 \]
	\bigskip
	\noindent
	\textbf{Solution:}
	
	\noindent
	Let $ a = 2 + \log x , \, b = -1 + \log x  , \, \text{and} \, c = 1 + \log x^2  $. Notice that $ a + b = c \Rightarrow \left( 2 + \log x \right) + \left( -1 + \log x \right) =\left( 1 + 2 \log x \right)  $. Thus,
		\begin{align*}
				{\left( 2 + \log x \right)}^3 + {\left( -1 + \log x \right)}^3 & = {\left( 1 + \log x^2 \right)}^3 \\
				a^3 + b^3 & = {(a+b)}^3 \\
				a^3 + b^3 & = a^3 + 3a^2b + 3ab^2 + b^3 \\
				0 & = 3ab \left( a+b \right)
		\end{align*}
	\medskip
		From this, we obtained $a = 0$,  $b = 0 $, and $ a+b = 0 $. \\
		If $ a = 0 $ we get,
		\[ 2 + \log x = 0 \Rightarrow x = \frac{1}{100}  \]
		If $ b = 0 $ we get,
		\[ -1 + \log x = 0 \Rightarrow {x = 10} \]
		If $ a + b = 0 $ we get,
		\[ 1 + 2 \log x = 0 \Rightarrow x = \frac{\sqrt{10}}{10}  \]
	
	\bigskip
	\noindent
	2. The positive integer $n$, when divided by \( 3, 4, 5, 6, \text{and} \, 7 \) leaves remainders of \(2, 3, 4, 5, \text{and} \, 6  \) respectively. Find the least possible value of $n$.
	
	\bigskip
	\noindent
	\textbf{Solution:} \\
	
	
	\noindent 
	Note that $n = 3a+2 = 4b+3 = 5c+4 = 6d+5 = 7e+6$. \\
	
	\noindent
	Adding $1$ to $n$ gives $n+1 = 3(a+1) = 4(b+1) = 5(c+1) = 6(d+1) = 7(e+1)$. Notice that $n+1$ is divisible by $3, 4, 5, 6, \text{and} \, 7 $. Therefore, the smallest positive number $n+1$ is given by 
	
	 \[ lcm(3, 4, 5, 6, 7) = 2^2 \times 3 \times 5 \times 6 \times 7 = 420   \]
	
	\noindent
	Hence, the smallest positive $n$ is $419$.
\end{document}
