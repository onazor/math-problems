\documentclass[14pt]{extreport}
\usepackage[letterpaper, margin=1in]{geometry}
\usepackage{setspace}
\usepackage[dvipsnames]{xcolor}
\usepackage{ragged2e}
\usepackage{amsmath}
\usepackage{cancel}
\usepackage{multicol}
\def\deg{\ensuremath{^\circ}}
\def\v#1{\ensuremath{\mathrm{#1}}}
\usepackage{multirow}
\usepackage{fancyhdr}
\pagestyle{fancy}
\usepackage{lastpage} 
\usepackage{colortbl}
\usepackage{amssymb}
\usepackage{graphicx}
\setlength{\parskip}{0pt}
\lhead{April 5, 2021}
\chead{}
\rhead{jnrozano@up.edu.ph}

\lfoot{Jhon Christian N. Rozano}
\cfoot{}
\rfoot{Page \thepage\ of \pageref{LastPage}}

\renewcommand{\headrulewidth}{1pt}

\renewcommand{\footrulewidth}{1pt}

\newcommand*\Eval[3]{\left.#1\right\rvert_{#2}^{#3}}

\definecolor{LightCyan}{rgb}{0.88,1,1}

\begin{document}
	\raggedcolumns
	\setlength{\columnsep}{10pt}
	\renewcommand{\arraystretch}{1.5}
	\singlespacing
	
	\begin{center}
		{\large  \textbf{Math Problem of the Day}} \\
		{\emph{Topic: Number Theory}} 
	\end{center}
	{ \LARGE
	\[ 9^{8^{7^{6^5}}}  \] }
	\noindent
	What are the last two digits when this integer is fully expanded out?
	
	\bigskip
	\noindent
	\textbf{Solution 1:} \\
	Observe that from repeated modular arithmetic,
	\begin{multicols}{2}
			\[ 9^1  \equiv \textbf{9} \pmod{100} \]
			\[ 9^2  \equiv \textbf{81} \pmod{100} \]
			\[	9^3  \equiv 9 \times 81 \equiv \textbf{29} \pmod{100} \]
			\[	9^4  \equiv 9 \times 29 \equiv \textbf{61} \pmod{100} \]
			\[	9^5  \equiv 9 \times 61 \equiv \textbf{49} \pmod{100} \]
			\[	9^6  \equiv 9 \times 49 \equiv \textbf{41} \pmod{100} \]
			\[	9^7  \equiv 9 \times 41 \equiv \textbf{69} \pmod{100} \]
			
	\columnbreak
			\[ 9^8  \equiv 9 \times 69 \equiv \textbf{21} \pmod{100} \]
			\[ 9^9  \equiv 9 \times 21 \equiv \textbf{89} \pmod{100} \] 
			\[ 9^{10}  \equiv 9 \times 89 \equiv \textbf{1} \pmod{100} \]
			\[ 9^{11}  \equiv 9 \times 1 \equiv \textbf{\textcolor{red}{9}} \pmod{100} \] 
			\[ 9^{12}  \equiv 9 \times 9 \equiv \textbf{\textcolor{red}{81}} \pmod{100} \]
			\[ 9^{13}  \equiv 9 \times 81 \equiv \textbf{\textcolor{red}{29}} \pmod{100} \]
			\[ \vdots \]
	\end{multicols}
	\noindent
	The cycle of $\pmod{100}$ repeats after 10 modular arithmetic expressions. In order to reduce the original expression, we need to find \( \displaystyle 8^{7^{6^5} }  \pmod{10} \). Thus,
	\[ \varphi(10) = 10 \left( 1 - \frac{1}{2} \right) \left(  1 - \frac{1}{5} \right) = 4 \]
	\large
	\[ 8^{{7^{6^5}}} \equiv 8^{{{(-1)}^{6^5} \pmod{4}}} \equiv 8^1 \Rightarrow 8 \]
	Therefore $8^{{7^{6^5}}} \equiv 8 \pmod{10}$.
	\[ 9^{8^{7^{6^5}}} \equiv 9^8 \equiv \fbox{\textcolor{red}{21}} \pmod{100}  \]
\end{document}
